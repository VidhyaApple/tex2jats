\begin{document}
\frenchspacing

\begin{frontmatter}



\title{A time-varying effect model for studying gender differences in health behavior}

%%begin{catchline}

%article-type=articles
%left-running-head=journal title
%right-running-head=author name
\newcommand{\artinfo}{\sffamily\fontsize{7pt}{8pt}\selectfont%
Statistical Methods in Medical Research\\
2015, Vol. 00(0) 00-00\\
\copyright\,The Author(s) 2014\\
Reprints and permission:\\ 
sagepub.co.uk/journalsPermissions.nav\\
DOI: 10.1177/0000000000000000\\
smm.sagepub.com\\
\includegraphics[scale=1]{SAGE_Logo.eps}
}
%%end{catchline}

\author{Songshan Yang\fnref{fn2}}
\author{James A. Cranford\fnref{fn3}}
\author{Runze Li\fnref{fn4}}
\author{Robert A. Zucker\fnref{fn3}}
\author{Anne Buu\fnref{fn1}%\cormark{cor1}
}

%\renewcommand{\baselinestretch}{1.0} \small\normalsize
\fntext[fn1]{School of Nursing, University of Michigan, Ann Arbor, MI 48109, U.S.A.}
\fntext[fn2]{Department of Statistics, Pennsylvania State University, University Park, PA 16802, U.S.A.}
\fntext[fn3]{Department of Psychiatry & Addiction Research Center, University of Michigan, Ann Arbor, MI 48109, U.S.A.}
\fntext[fn4]{Department of Statistics and The Methodology Center, Pennsylvania State University, University Park, PA 16802, U.S.A.}
%\renewcommand{\baselinestretch}{2.0} \small\normalsize
\cortext[cor1]{Address for correspondence: Anne Buu, School of Nursing, University of Michigan, 400 North Ingalls, Ann
Arbor, MI 48109, USA.}
\cortextemail[cor1]{E-mail: buu@umich.edu}

\begin{abstract}
This a\textbf{rtic}le proposes a time-varying effect model that can be used
to characterize gender-specific trajectories of health behaviors
and conduct hypothesis testing for gender differences. The
motivating examples demonstrate that the proposed model is
applicable to not only multi-wave longitudinal studies but also
short-term studies that involve intensive data collection. The
simulation study shows that the accuracy of estimation of
trajectory functions improves as the sample size and the number of
time points increase. In terms of the performance of the 
hypothesis testing, the type I error rates are close to their
corresponding significance levels under all combinations of sample
size and number of time points. Furthermore, the power increases
as the alternative hypothesis deviates more from the null
hypothesis and the rate of this increasing trend is higher when
the sample size and the number of time points are larger.
\end{abstract}

\begin{keyword}
longitudinal data; time-varying effect; mixed effect; B-spline; substance abuse.
\end{keyword}
\end{frontmatter}

\section{Introduction} 
Historically, in many fields of health
science such as alcohol and substance abuse research, participants
were largely male. Thus, our knowledge of the relevant diseases
has been built upon data from predominantly male samples. However,
females and males are different not only biologically but also
psychosocially. The complex interplay between these two types of
factors at different developmental stages may result in gender
differences that vary across time \cite{buu14,buu15}. For example,
females' earlier timing of physical puberty may put them at higher
risk for alcohol use in early adolescence \cite{wichstrom01}.
Other biological factors may, however, prevent females from
drinking alcohol in later developmental stages such as higher
reactivity to alcohol and greater vulnerability to adverse health
effects due to alcohol use \cite{nolen04}. In addition,
psychosocial factors such as greater social sanctions against
females' alcohol use may contribute to gender differences
\cite{nolen04}. In fact, the finding of a closing gender gap in
alcohol use, abuse, and dependence in the United States population
\cite{keyes08} tends to reflect the diminishing gender-based
drinking norms in the modern society. By understanding gender
differences in developmental trajectories of alcohol use or
drinking patterns, we may be able to tailor prevention and
intervention efforts to the special timing and risky patterns of
each gender group and, thus, thwart progression to worse outcomes.

According to our review of the empirical studies published in the
past two decades, gender has been conventionally treated as a
time-invariant effect, that is, both the magnitude and direction
of gender differences remain unchanged over time (e.g.
\cite{Boden14,smith15}). Yet, a recent study \cite{chen12}
analyzed 4 waves of data from a nationally representative sample
of adolescents and found similar developmental trajectories for
alcohol use, nicotine use and marijuana use by gender: females
tended to be involved in higher levels of substance use in early
adolescence, whereas males exhibited greater increases across time
and higher levels of use in mid-adolescence and early adulthood.
An important limitation of the methodology used in the study is
that the developmental trajectories for all the three substances
were derived by fitting quadratic growth models that pre-specified
a simple shape for developmental changes, which may contribute to
the similarity in trajectories across substances. In addition,
gender differences were tested \emph{implicitly} through the
interactions between the group indicator and linear/quadratic
terms. Such simple shapes can hardly characterize the complex
developmental trajectories of substance use, especially when there
are many time points spanning a long developmental period.

In recent years, the time-varying effect model (TVEM) has been
introduced and extended for broader applications in the substance
abuse field \cite{cai00,dziak14,qu06,tan12,wang09,zhu12}. The TVEM
\emph{explicitly} characterizes gender differences in
developmental trajectories of substance use across the critical
period by modeling gender as a time-varying effect, which can be
easily understandable to the scientific community as well as the
general public through graphical presentation. Such trajectories
are estimated through nonparametric regression functions that do
not assume fixed shapes like conventional growth curves, and thus
may reveal different timing of prevention and intervention for
gender groups. Furthermore, this kind of model can be applied to
not only multi-wave longitudinal studies, but also short-term
studies that involve intensive data collection methods such as
ecological momentary assessment or daily diaries, which are
becoming more popular for studying patterns of health behaviors
\cite{liu13,selya13,vasilenko14,yang15}. Although these two types
of studies are different in terms of the duration between time
points, the numbers of time points and the richness of data in
both studies tend to be beyond what traditional parametric methods
can fully characterize.

This study proposes a time-varying effect model to (1)
characterize gender-specific trajectories of health behaviors and
(2) conduct hypothesis testing for gender differences. We
demonstrate an application of 
using longitudinal data
from a well-known prospective study on the development of
substance use and abuse, the Michigan Longitudinal Study
(MLS)\cite{zucker96}, that is following high risk youth from
childhood to adulthood. Another application of the model is
demonstrated by using daily diary data from the National Survey of
Midlife Development in the United States (MIDUS) \cite{ryff10}.
The MIDUS data were also used to design a simulation study
evaluating the performance of the proposed method in terms of the
accuracy of estimation of trajectories, type I error rate, and
statistical power under different conditions.


\section{The model}

Let $Y(t_{ij})$ be the $j$-th observed outcome from the $i$-th
subject at $t_{ij} (i=1,......,N; j=1,......,J_i)$ and $k$ be the
group (e.g. gender) that Subject $i$ belongs to ($k=1,2$). We
consider the following nonparametric generalized mixed effect
model for the measurement:

\begin{equation}\label{Model}
g[E\{Y(t_{ij})|b_{i}\}]=\mu(t_{ij})+\beta(t_{ij})I_{\{k=1\}}+b_i,
\end{equation}
where $g(\cdot)$ is a known link function under the framework of
generalized linear models \cite{mccullagh89}; $\mu(t_{ij})$ is the
trajectory of the $k=2$ group; $\beta(t_{ij})$ delineates the
time-varying difference between the two groups; and $b_i$ is a
random individual effect modeling within-subject correlation and
is assumed to follow a normal distribution with mean 0 and
variance $\sigma^2$. The distribution of $Y(t_{ij})$ is thus
completely characterized by $\mu(t_{ij})$, $\beta(t_{ij})$ and
$b_i$. One advantage of this model over fitting a separate
nonparametric trajectory for each group is that $\beta(t_{ij})$
can characterize the group difference across time explicitly.


The time-varying coefficients in Model (\ref{Model}) can be
represented using basis expansions. Thus, the nonparametric
functions $\mu(t_{ij})$, $\beta(t_{ij})$ are treated as linear
combinations of several known parametric functions. We choose to
use a spline basis to represent them as piecewise cubic functions.
On each of several intervals defined by knots, the spline function
is cubic. At a knot, the spline function is continuous and has
continuous first and second derivatives, although the third
derivative may be discontinuous at the knots. This allows any
smooth shape to be approximated well if enough knots are used.
Specifically, we use a B-spline basis \cite{deboor78}, which can
be automatically generated by most commonly used statistical
software packages such as SAS and R, for a given set of knots. The
basis functions are always nonnegative, and each is zero over most
of the interval, so that each knot's basis function is orthogonal
to the other basis functions except for its closest knot
neighbors. However, overfitting may still occur with a B-spline
basis if too many knots are used. For simplicity, in this paper,
we adopt the approach of \textbf{\cite{shiyko12}} that used a
small number of equally spaced knots and treated the selection of
the number of basis functions as a model selection problem.


After defining the basis functions using the B-spline formula,
Equation (\ref{Model}) can be written as a parametric form:
\begin{equation}\label{B-spline}
g[E\{Y(t_{ij})|b_{i}\}]=\Sigma_{p=1}^{P}\xi_{p}\phi_{p}(t_{ij})+\Sigma_{q=1}^{Q}\zeta_{q}\psi_{q}(t_{ij})I_{\{k=1\}}+b_i,
\end{equation}
where $\phi_{1}(t_{ij}),\ldots,\phi_{P}(t_{ij})$ and
$\psi_{1}(t_{ij}),\ldots,\psi_{Q}(t_{ij})$ are known functions of
time defined using the recursive B-spline formulas;
$\xi_1,\ldots,\xi_P$ and $\zeta_1,\ldots,\zeta_Q$ are the
corresponding regression coefficients. Equation (\ref{B-spline})
is thus a generalized linear mixed model with $P+Q+1$ parameters.
For the binary case, the likelihood function for Subject $i$ is
$$
L_{i}=\int\prod_{j=1}^{J_{i}}
p_{ij}^{Y(t_{ij})}(1-p_{ij})^{(1-Y(t_{ij}))}
\exp(-b_i^2/2\sigma^2)/\sigma\,d b_i,
%L_{i}=\int\prod_{j=1}^{J_{i}}P(Y_{ij}=0)^{I_{\{Y_{ij}=0\}}}P(Y_{ij}=1)^{1-I_{\{Y_{ij}=0\}}}dF(b_{i},\sigma^{2}),
$$
where $p_{ij}=P(Y(t_{ij})=1|b_i)$.
%is the indicator function and $F(b_{i},\sigma^{2})$ is
%the distribution function of $b_{i}$.
For the continuous case, it is
$$
L_{i}=\int\prod_{j=1}^{J_{i}}f(Y(t_{ij}))\exp(-b_i^2/2\sigma^2)/\sigma\,d
b_i,
$$
where $f(y(t_{ij}))$ stands for the density function of
$Y(t_{ij})$. The likelihood function for the entire sample is thus
$L=\prod_{i=1}^{N}L_{i}$. We use the R-package lme4 to derive the
maximum likelihood estimates of the parameters.

In practice, researchers are interested in graphing
gender-specific trajectories. Based on the fixed effects in
Equation (\ref{B-spline}), the trajectory for the $k=2$ group is
$\hat{f}_{2}(t_{ij})=\mu(t_{ij})=\Sigma_{p=1}^{P}\xi_{p}\phi_{p}(t_{ij})$.
For the $k=1$ group, the trajectory is
$\hat{f}_{1}(t_{ij})=\mu(t_{ij})+\beta(t_{ij})=\Sigma_{p=1}^{P}\xi_{p}\phi_{p}(t_{ij})+\Sigma_{q=1}^{Q}\zeta_{q}\psi_{q}(t_{ij})$.
The delta method can be used to estimate the variance of the
estimated functions of these two groups at any time $t$. In this
way, we can obtain the pointwise confidence intervals which can be
plotted along with the estimated trajectories.

Researchers are also interested in testing whether there exists
any gender difference. We formulate this hypothesis testing
problem as follows:
\begin{center}
$H_{0}: \beta(t)=0$ v.s. $H_{1}: \beta(t)\neq0$
\end{center}
Under $H_{0}$, the two groups have the same trajectory (i.e.,
there is no group difference). Following the method described
above, we can estimate $\mu(t)$ under $H_{0}$, as well as $\mu(t)$
and $\beta(t)$ under $H_{1}$. We can further evaluate the
log-likelihood functions under $H_{0}$ and $H_{1}$ denoted by
$\ell(H_{0})$ and $\ell(H_{1})$, respectively. The generalized
likelihood ratio test(GLRT) for the hypothesis can thus be defined
by $T=2\{\ell(H_{1})-\ell(H_{0})\}$. Following \cite{cai00}, we
can conduct bootstrap sampling to estimate the p-value for the
GLRT.


\section{The motivating examples}

We use two well-known studies in the field of health behavior
research to demonstrate that the proposed model can be applied to
not only multi-wave longitudinal studies, but also short-term
studies that involve intensive data collection such as daily
process studies.



\newpage
\begin{table}[htbp]
\footnotesize
\begin{center}
\caption{MISE under different sample sizes, numbers of time
points, proportions of zeros, and gender ratios.}
%\scalebox{0.88}{%
\begin{tabular}{c|cc|cc|cc}
 
%\begin{thead}
&\multicolumn{6}{c}{male:female=1:1}\\ 
& \multicolumn{2}{c|}{$N=100$}& \multicolumn{2}{c|}{$N=200$}& \multicolumn{2}{c}{$N=400$}\\
&$J=8$&$J=16$&$J=8$&$J=16$&$J=8$&$J=16$\\
 Proportion & Mean(SD) & Mean(SD) & Mean(SD) & Mean(SD) &
Mean(SD) & Mean(SD)\\
 %\end{thead}
 30\%&0.387(0.273) &
0.179(0.112)&0.171(0.106) &0.086(0.048) &0.082(0.047) &
0.044(0.023)       \\ 
50\% & 0.287(0.183)        & 0.148(0.077)         &  0.140(0.080)        &   0.078(0.043)       &   0.074(0.038)       & 0.042(0.023)         \\
70\% &  0.557(0.444)       &   0.306(0.426)       &  0.294(0.323)        &  0.140(0.135)        &  0.156(0.176)        & 0.071(0.051)    \\
 &\multicolumn{6}{c}{male:female=3:1}\\ 
& \multicolumn{2}{c|}{$N=100$}& \multicolumn{2}{c|}{$N=200$}& \multicolumn{2}{c}{$N=400$}\\

&$J=8$&$J=16$&$J=8$&$J=16$&$J=8$&$J=16$\\
 Proportion & Mean(SD) & Mean(SD) & Mean(SD) & Mean(SD) &
Mean(SD) & Mean(SD)\\
 
 30\%&0.417(0.236) &0.260(0.401)&0.231(0.188) &0.104(0.067) &0.099(0.059) &0.050(0.028)       \\
50\% & 0.319(0.179)        & 0.151(0.081)         &  0.153(0.086)        &   0.078(0.039)       &   0.077(0.039)       & 0.042(0.022)         \\
70\% &  0.453(0.263)       &   0.266(0.241)       &  0.288(0.269)        &  0.131(0.102)        &  0.126(0.095)        & 0.063(0.041)    \\

\end{tabular}
%}
%\footnotesize or source{footnote}
\end{center}
\normalsize
\end{table}




\begin{table}[htbp]
\begin{center}
\caption{Type I error rates under different sample sizes and
numbers of time points.}
\begin{tabular}{c|cccccc}
%\begin{thead}  
 &  \multicolumn{2}{c}{N=100}  & \multicolumn{2}{c}{N=200} & \multicolumn{2}{c}{N=400} \\
& J=8& J=16&  J=8&J=16 &J=8&J=16\\
%\end{thead}     
$\alpha=0.01$&0.011 &0.008 &0.01  &0.006 &0.006&0.016\\
$\alpha=0.05$&0.053 &0.046 &0.034  &0.059 &0.057&0.048\\
$\alpha=0.10$&0.114 & 0.088& 0.097 &0.108 &0.107&0.108\\
$\alpha=0.25$&0.286 & 0.26& 0.261 &0.231 &0.266&0.234\\
      
\end{tabular}
\end{center}
\end{table}



\newpage
\begin{figure}
  \begin{center}
    \includegraphics[width=\textwidth]{images/national_and_mls.pdf}
    \caption{Gender-specific trajectories based on the motivating examples.}
  \end{center}
\end{figure}

\begin{figure}
\centering
\includegraphics[width=0.45\textwidth]{images/n=100_j=8.pdf}
\includegraphics[width=0.45\textwidth]{images/n=100_j=16.pdf}
\includegraphics[width=0.45\textwidth]{images/n=200_j=8.pdf}
\includegraphics[width=0.45\textwidth]{images/n=200_j=16.pdf}
\includegraphics[width=0.45\textwidth]{images/n=400_j=8.pdf}
\includegraphics[width=0.45\textwidth]{images/n=400_j=16.pdf}
\caption{Power curves under different sample sizes, numbers of
time points, and significance levels.}
\end{figure}



\subsection{The Michigan Longitudinal Study (MLS)}

The MLS is an ongoing prospective study of people at high risk for
substance abuse and disorder \cite{zucker96}. It is the
developmentally earliest study currently extant and is also one of
the longest running projects in the field of substance abuse. We
chose to use data from the MLS to demonstrate the application of
the proposed method, because the study is highly influential and
the features of the data are typical in the field. Thus, our
methodological work may have high applicability to the field. More
importantly, these rich longitudinal data provide a rare
opportunity for us to examine gender differences in alcohol use
developmentally. Such investigation could shed some light on
future prevention and intervention work.

The MLS recruited participant families using fathers' drunk
driving conviction records and door-to-door community canvassing
in a four-county area in mid-Michigan. All participants received
extensive in-home assessments of their substance use and related
risk factors and consequences at baseline, and thereafter at
3-year intervals. The children of participant families were
followed from early childhood to adulthood. During the critical
developmental period of alcohol use onset and peak use (early
adolescence to young adulthood), these children were assessed
annually in order to measure drinking onset and patterns more
accurately. In this study, we use longitudinal data (ages 12 to
26) from a sample of 699 children (70.2\% males) for analysis. The
maximum number of time points available is 15, although some
participants may skip certain time points.

Our investigation aims to (1) characterize gender-specific alcohol
use behavior developmentally from early adolescence to young
adulthood, and (2) test gender differences in developmental
trajectories. In our analysis, the outcome at each time point is a
composite measure of alcohol consumption which is the product of
the number of drinking days in past month and the average number
of drinks per drinking day. Gender is treated as a time-varying
effect through $\beta(t)$ in Model (\ref{Model}); the link
function $g(\cdot)$ is the identity. Using AIC and BIC, we choose
5 knots to approximate the trajectories. Panels (a) and (b) in
Figure 1 show the developmental trajectories of females and males,
respectively. Using the asymptotic normality of the resulting
estimate, the asymptotic pointwise confidence intervals (CI) of
the trajectories of group 1 and group 2 (i.e., $\mu(t)+\beta(t)$
and $\mu(t)$) at time $t_{ij}$ are
$$
\hat{\mu}(t_{ij}) + \hat{\beta}(t_{ij})\pm
z_{\frac{\alpha}{2}}\sqrt{(\Phi(t_{ij}),\Psi(t_{ij}))\Sigma_{\xi,\zeta}(\Phi(t_{ij}),\Psi(t_{ij}))^{T}}$$
and
$$
\hat{\mu}(t_{ij})\pm
z_{\frac{\alpha}{2}}\sqrt{\Phi(t_{ij})\Sigma_{\xi}\Phi^{T}(t_{ij})},
$$
respectively, where $\Sigma_{\xi}$ and $\Sigma_{\xi,\zeta}$ are
the covariance matrices of the estimates of the covariates; $\Phi$
and $\Psi$ are the Bspline basis used to fit the functions
$\mu(t)$ and $\beta(t)$, respectively. The two gender groups do
not differ until middle adolescence. Although alcohol consumption
for both groups increases from middle adolescence to young
adulthood, the rate of change is much higher among males.
Furthermore, the decreasing trend for both groups after age 24
probably reflects that people tend to ``mature out" of heavy
drinking due to family or job responsibilities. Moreover, the
result of the generalized likelihood ratio test indicates that
there are significant gender differences with $T=161.29$ and the
corresponding p-value estimated to be around $0$ by bootstrap
sampling.

\subsection{The National Survey of Midlife Development in the
United States (MIDUS II)}

The Daily Stress Project of MIDUS II was designed to examine how
sociodemographic factors, health status, personality
characteristics, and genetic endowment modify patterns of change
in exposure to day-to-day life stressors as well as physical and
emotional reactivity to these stressors \cite{ryff10}. The survey
data were collected from $2,022$ noninstitutionalized adults aged
35 to 85 in the United States from 2004 to 2009. Participants
completed daily phone surveys on stressors, physical symptoms, and
positive/negative affect for 8 consecutive days.

Although daily process methods provide prospective data for
examining the dynamic associations between health behaviors, they
unavoidably involve self-monitoring of the target behavior, which
is an active component of some cognitive-behavioral interventions
\cite{simpson05}. The potential \emph{measurement reactivity}
(defined changing the target behavior due to self-awareness) is
undesirable for those studies that aim to investigate the
association between the target behavior and its precursor or
consequence. On the other hand, for those applications aiming to
facilitate behavior changes, such an effect can be used to boost
or extend intervention effects \cite{tucker12}. Thus, verifying
measurement reactivity is an important research question,
especially given that existing empirical investigations are few
and have produced mixed results \cite{barta12,stritzke05}.

Our investigation aims to (1) characterize the gender-specific
patterns of change in self-reported physical symptoms during the 8
days of assessment, and (2) test gender differences in patterns of
change. In our analysis, the outcome at each time point is a
binary variable for any physical symptoms reported in a day
($1=$yes; $0=$no). Gender is treated as a time-varying effect
through $\beta(t)$ in Model (\ref{Model}); the link function
$g(\cdot)$ is the logit. Using AIC and BIC, we choose 3 knots to
approximate the patterns of change. Panels (c) and (d) in Figure 1
show the patterns of change in the odds of reporting any physical
symptoms for females and males, respectively. Females tend to have
much greater odds of reporting physical symptoms, particularly in
the first few days. Further, measurement reactivity is evident in
both gender groups in the beginning of the assessment period.
Moreover, the result of the generalized likelihood ratio test
indicates that there are significant gender differences with
$T=50.42$ and the corresponding p-value estimated to be about $0$
by bootstrap sampling.


\section{Simulation study}

In this section, we examine the performance of the proposed method
under different situations. The design of the simulation is based
on the feature of the MIDUS data demonstrated in the motivating
examples section. The response is a binary variable which follows
Model (\ref{Model}) with the link function $g(\cdot)$ being the
logit function and the group indicator $k=1$ corresponding to
females. We generate the response $Y$ using the estimated
functions of $\mu(t)$, $\beta(t)$ and $b_i$ from the fitted model
of these national data. We manipulate \textbf{four} factors: (i)
the sample size: $N=100, 200$ and $400$; (ii) the number of time
points: $J=8$ and $16$; (iii) the proportion of zeros in the
response $Y$ from all participants across all waves: 30\%, 50\%,
and 70\%; and (iv) the gender ratio: $1:1$ and $3:1$. We expect
the performance improves with the sample size and the number of
time points. We also expect that the proposed method may perform
better in the setting with $50\%$ zeros than in the other
settings, because the former tends to have greater Fisher
information. Furthermore, we expect the performance to be better
when the sample sizes of the two gender groups are about the same.

\subsection{The Accuracy of Estimation of Trajectory Functions}

We evaluate the performance of the proposed method concerning the
accuracy of the estimation of trajectory functions, $\mu(t)$ and
$\beta(t)$, under different combinations of the sample size, the
number of time points and the proportion of zeros. It is
straightforward to manipulate the former two factors. However, the
proportion of zeros in the response is manipulated indirectly by
altering $\mu(t)$ and $\beta(t)$. Let $\hat{\mu}_{0}(t)$ and
$\hat{\beta}_{0}(t)$ be the trajectory functions in the model
fitted on the national data. We obtain three different proportions
of zeros, $30\%$, $50\%$, and $70\%$, through the following
trajectory functions, respectively: (1) $\mu(t)=\hat{\mu}_{0}(t)$
and $\beta(t)=\hat{\beta}_{0}(t)$; (2) $\mu(t)=\hat{\mu}_{0}(t)$
and $\beta(t)=-2\hat{\beta}_{0}(t)$; and (3)
$\mu(t)=-1.8\hat{\mu}_{0}(t)$ and $\beta(t)=\hat{\beta}_{0}(t)$.
Please note that the function $\mu(t)$ here refers to the one
defined in Equation (1) rather than the mean of the response $Y$.
The criterion for evaluating the accuracy of estimation of
trajectory functions is the mean integrated squared error (MISE):
\begin{center}
$MISE=\frac{1}{NJ}\Sigma_{i=1}^{N}\Sigma_{j=1}^{J}((\hat{\mu}(t_{ij})-\mu(t_{ij}))^{2}+(\hat{\beta}(t_{ij})-\beta(t_{ij}))^{2}).$
\end{center}
Table 1 shows the MISE and its empirical standard error under
different combinations of the three factors. Based on the table,
the accuracy of the proposed method improves as the sample size
and the number of time points increase. For example, when $N=100$
and $J=8$, the MISE and the corresponding standard error are
larger in comparison to other settings with the same proportion of
zeros in the responses. Controlling for the effects of sample size
and number of time points, when the proportion of zeros increases
from 30\% to 50\%, the performance of our method does not change
much as demonstrated by the similar MISEs and standard errors.
However, when the proportion of zeros increases from 50\% to 70\%,
the performance of our method becomes worse based on the larger
MISEs and standard errors. Furthermore, the performance of the
method tends to be better when the gender ratio is $1:1$ (like the
national data example), in comparison to the setting when the
gender ratio is $3:1$ (like the MLS data example). However, the
impact of gender ratio becomes unimportant when the proportion of
zeros is 70\% and the performance is poor anyway.

\subsection{The Type I Error Rate and Power of the Hypothesis
Testing}

We also evaluate the performance of the hypothesis testing in
terms of the type I error rate and power. Our simulation involves
the setting of $H_{0}: \beta(t)=0$ versus $H_{1}:
\beta(t)=\delta\hat{\beta}(t)$, where $\hat{\beta}(t)$ is the
trajectory function in the model fitted on the national data; and
the value of $\delta$ is manipulated to reflect different levels
of deviation from $H_0$. In this part of the simulation study, we
only manipulate the sample size and the number of time points
because manipulating the proportion of zeros would require
altering $\beta(t)$. We also adopt the gender ratio in the
national data when generating the gender of the subjects in the
simulation samples. Because the proposed model is nonparametric,
we can only derive the empirical distribution of the test
statistics $T=2\{\ell(H_{1})-\ell(H_{0})\}$ through simulation.
Under each combination of the sample size and the number of time
points, we generate $10000$ data sets under $H_0$ and then conduct
the hypothesis testing which results in $10000$ values of $T$. The
critical values of $T_{0.01}, T_{0.05}, T_{0.10}, T_{0.25}$ are
thus the $99$th, $95$th, $90$th, and $75$th percentiles from this
empirical distribution of $T$. After obtaining the critical values
under each situation, we investigate the effect of $\delta$ on the
power of the test by taking a grid of $\delta$ over $(0,2)$. Under
each situation, we generate $1000$ data sets for each value of
$\delta$ and conduct the hypothesis testing on the data sets
(using the 4 critical values) to examine the type I error rate and
the power corresponding to the 4 values of $\alpha$ (i.e.
significance levels). Table 2 shows the type I error rates (i.e.
the values of power when $\delta=0$), which are close to their
corresponding significance levels under all settings. Figure 2
depicts the power as a function of $\delta$ and $\alpha$ under
different situations. As demonstrated by the figure, the power
increases as $\delta$ increases and the value of power is 1 when
$\delta$ is greater than 1.5. In addition, the rate of this
increasing trend is higher when the sample size and the number of
time points are larger.

\section{Discussion}

This study proposes a time-varying effect model that can be used
to characterize gender-specific trajectories of health behaviors
and conduct hypothesis testing for gender differences. The major
strengths of the model include (1) treating gender differences as
a time-varying effect; and (2) modeling time-varying effects with
flexible smooth functions derived from empirical data. The
proposed model can be applied to not only multi-wave longitudinal
studies like the MLS, but also short-term studies that involve
intensive data collection such as the daily diary data from the
MIDUS. Furthermore, the design of our simulation study is unique
because it simulates the features of the MIDUS data so that the
results can be generalizable to the field of health behavior
research.

The simulation study shows that the accuracy of estimation of
trajectory functions improves as the sample size and the number of
time points increase. Controlling for the effects of these two
factors, the proportion of zeros only has a considerable negative
effect on accuracy when it increases from 50\% to 70\%. In terms
of the performance of the hypothesis testing, the type I error
rates are close to their corresponding significance levels under
all combinations of sample size and number of time points.
Furthermore, the power increases as the alternative hypothesis
deviates more from the null hypothesis and the rate of this
increasing trend is higher when the sample size and the number of
time points are larger.

Although the methodology proposed in this study was motivated by
our research interest in gender differences, it can be applied to
a variety of contexts that involve the comparison between two
trajectories or change patterns. For example, the model can be
used to characterize the developmental trajectory of substance use
among children of alcoholic (COA) and compare it with the
trajectory of non-COA. Future work may be needed to extend the
methodology to handle the settings with more than two groups such
as studying racial differences.

The proposed model accounts for subject-specific effects only
through a random intercept. Thus, the outcomes of the same subject
at any two time points are treated as equally correlated
regardless of the size of the time difference between the time
points, and this may not be realistic in some settings. Popular
longitudinal models assuming autocorrelation for the residuals
(such as AR-1), on the other hand, treat the lag between each
consecutive pair of measurements as equivalent. This assumption
may not apply to those settings that involve random or
inconsistent measurement times. A more straightforward way to
incorporate longitudinal correlation, therefore, is to add a
random subject-level slope to the right-hand side of Model (1).
Further research is, thus, warranted regarding how best to
implement more complex random effect structures while still
allowing the time-varying effects to be easily estimated and
interpreted with standard software.


In spite of the fact that the proposed model can handle a variety
of measurement scales under the framework of generalized linear
models such as continuous and binary outcomes (as demonstrated in
our motivating examples), future work is needed to extend the
model to deal with other scales that are also common in the health
behavior field including ordinal outcomes (e.g. \cite{dziak14})
and zero-inflated counts (e.g. \cite{buu11,buu12}). Furthermore,
our work in this paper focuses on the setting that involves a
single health behavior. Future studies may consider modeling
\emph{multiple} health behaviors that tend to co-occur such as
substance use behaviors, violence behaviors, and HIV sexual risk
behaviors.

%extra content to check


\begin{enumerate*}
\item \textbf{Multiplication of two numbers with the same base and different exponents.} Add the exponents together:\vspace*{-3pt}
\begin{equation*}
x^a x^b=x^{a+b}.
\end{equation*}
\item \textbf{Division of two numbers with the same base and different exponents.} Subtract the bottom exponent from the top exponent:
\begin{equation*}
\frac{x^a}{x^b}=x^{a-b}.
\end{equation*}
\item \textbf{Negative exponents.}  Take the reciprocal, but change the negative exponent to a positive one:\vspace*{-3pt}
\begin{equation*}
x^{-a}=\frac{1}{x^a}.
\end{equation*}
\item \textbf{Exponents of 0.}  $0^0$ is undefined, but taking any other number to the zero power produces 1:\vspace*{-3pt}
\begin{equation*}
x^0=1 \text{   for any   }x\neq 0.
\end{equation*}
\item \textbf{Exponents of 1.}  Taking the first power of any number returns the number:
\begin{equation*}
x^1=x\text{   for all   }x.
\end{equation*}
\item \textbf{Multiple layers of exponents.}  Multiply the different layers together:
\begin{equation*}
(x^a)^b=x^{ab}.
\end{equation*}
\item \textbf{Multiplication of numbers with different bases and the same exponent.}  The product of the different bases can be set to the common exponent:
\begin{equation*}
x^ay^a=(xy)^a.
\end{equation*}
\item \textbf{Division of numbers with different bases and the same exponent.} The fraction of the different bases can be set to the common exponent:
\begin{equation*}
\frac{x^a}{y^a}=\bigg(\frac{x}{y}\bigg)^a.
\end{equation*}
\end{enumerate*}

\begin{enumerate*}
\item $_a\sqrt{x}_b\sqrt{x}=x^{\frac{1}{a}}x^{\frac{1}{b}}=x^{\frac{1}{a}+\frac{1}{b}}=x^{\frac{a+b}{ab}}=({x^{a+b}})^{\frac{1}{ab}}=_{ab}\sqrt{x^{a+b}}.$\\[4pt]
\item $\frac{_a\sqrt{x}}{_b\sqrt{x}}=\frac{x^{\frac{1}{a}}}{x^{\frac{1}{b}}}=x^{\frac{1}{a}-\frac{1}{b}}=x^{\frac{b-a}{ab}}=({x^{b-a}})^{\frac{1}{ab}}=_{ab}\sqrt{x^{b-a}}.$\\[4pt]
\item $_{-a}\sqrt{x}=x^{-\frac{1}{a}}=\frac{1}{x^{\frac{1}{a}}}=\frac{1}{_a\sqrt{x}}.$\\[4pt]
\item $_1\sqrt{x}=x$ for all $x$, so we never consider these \textit{first} roots.\\[4pt]
\item $_b\sqrt{_a\sqrt{x}}=(x^\frac{1}{a})^\frac{1}{b}=x^{\frac{1}{ab}}=_{ab}\sqrt{x}$.\\[4pt]
\item $_a\sqrt{x}_a\sqrt{y}=x^{\frac{1}{a}}y^{\frac{1}{a}}=(xy)^{\frac{1}{a}}=_a\sqrt{xy}$.
\end{enumerate*}

\begin{itemize*}
\item \textbf{Step 1.} Find the LCD.  Here LCD$(4,6)=12$.
\item \textbf{Step 2.} For each fraction, multiply the numerator and denominator by the number that makes the denominator equal to the LCD.  For the first fraction, that number is 3 since $4\times3=12$, and for the second fraction that number is 2.  So, changing the fractions, we obtain
\begin{equation*}
\frac{3\times3}{4\times3}+\frac{5\times2}{6\times2}=\frac{9}{12}+\frac{10}{12}.
\end{equation*}
\item \textbf{Step 3.} Add the numerators, and leave the common denominator alone:
\begin{equation*}
\frac{9}{12}+\frac{10}{12}=\frac{19}{12}.
\end{equation*}
\end{itemize*}

\begin{enumerate*}
\setlength\itemsep{3pt}
\item[(a)] How many possible AES-128 keys are there? Write your answer with an exponent, but in addition, \deleted{in order}to have a sense of how large this number is, use Stata, R, or a similar program to write out the number approximately. (\textit{Hint}: The notation ``e+10'' means that you should move the decimal point 10 places to the right, so 1.23e+10 is 12,300,000,000.) 
\item[(b)] Suppose that a supercomputer can check 1,000,000,000,000,000 (1 quadril\-lion) possible AES-128 keys every second.  Approximately how long (\textit{hint}: in \textit{years}) would it take the computer to test every possible AES-128 key? 
\item[(c)] In theory, but probably not yet in reality, there can exist something called a quantum computer that uses the laws of quantum physics to perform many computations at once. 
\end{enumerate*}

\begin{quotation}
\noindent Consider a problem that has these four properties:
\begin{enumerate*}
\item[1.] The only way to solve it is to guess answers repeatedly and check them,
\item[2.] the number of possible answers to check is the same as the number of inputs,
\item[3.] every possible answer takes the same amount of time to check, and
\item[4.] there are no clues about which answers might be better: generating possibilities randomly is just as good as checking them in some special order.
\end{enumerate*}\vspace{3pt}

\end{quotation}

%back-start
\section*{Acknowledgements}
The work of Yang and Li was supported by the National Institutes
of Health (NIH) [P50 DA010075, P50 DA036107 & R01 CA168676] and
the National Science Foundation (NSF) [DMS 1512422]; The work of
Cranford and Buu was supported by the NIH [R01 DA035183]; and
Zucker's research was supported by the NIH [R01 AA07065]; and The
content is solely the responsibility of the authors and does not
necessarily represent the official views of the NIH and NSF.
\section*{Conflict of interest statements}
The authors declare that there is no conflict of interest.

\begin{thebibliography}
     \bibitem{buu14} Buu A, Dabrowska A, Mygrants M, et al. Gender differences in
the developmental risk for onset of alcohol, nicotine, and
marijuana use and the effects of nicotine and marijuana use on
alcohol outcomes. \textit{Journal of Studies on Alcohol and Drugs}
2014; 75: 850-858.
     \bibitem{buu15} Buu A, Dabrowska A, Heinze JE, et al. Gender differences in the developmental
trajectories of multiple substance use and the effect of nicotine
and marijuana use on heavy drinking in a high risk sample.
\textit{Addictive Behaviors} 2015; 50: 6-12.
     \bibitem{wichstrom01} Wichstrom L. The impact of pubertal
timing on adolescents' alcohol use. \textit{Journal of Research on
Adolescence} 2001; 11: 131-150.
     \bibitem{nolen04} Nolen-Hoeksema S. Gender differences in
     risk factors and consequences for alcohol use and problems. \textit{Clinical Psychology Review} 2004; 24: 981-1010.
     \bibitem{keyes08} Keyes KM, Grant BF and Hasin DS. Evidence for a closing gender
     gap in alcohol use, abuse, and dependence in the United States population. \textit{Drug and Alcohol Dependence} 2008; 93: 21-29.
      \bibitem{Boden14} Boden JM, Fergusson DM and Horwood LJ. Associations between exposure to stressful life events and alcohol use disorder
      in a longitudinal birth cohort studied to age 30. \textit{Drug and Alcohol Dependence} 2014; 142: 154-160.
       \bibitem{smith15} Smith PH, Kasza KA, Hyland A, et al. Gender differences in medication use and cigarette smoking cessation:
       results from the international tobacco control four country survey. \textit{Nicotine & Tobacco Research} 2015; 17: 463-472.
     \bibitem{chen12} Chen P and Jacobson KC. Developmental trajectories
     of substance use from early adolescence to young adulthood: Gender and racial/ethnic differences. \textit{Journal of Adolescent Health} 2012; 50: 154-163.
     \bibitem{cai00} Cai Z, Fan J and Li R. Efficient estimation
     and inferences for varying-coefficient models. \textit{Journal of American Statistical Association} 2000; 95: 888-902.
     \bibitem{dziak14} Dziak JJ, Li R, Zimmerman MA, et al. Time-varying effect models for ordinal responses with applications
in substance abuse research. \textit{Statistics in Medicine} 2014;
33: 5126-5137.
     \bibitem{qu06} Qu AP and Li R. Quadratic inference functions for
     varying-coefficient models with longitudinal data. \textit{Biometrics} 2006; 62: 379-391.
     \bibitem{tan12} Tan X, Shiyko M, Li R, et al. A time varying effect model for Intensive longitudinal data. \textit{Psychological Methods} 2012;
     17: 61-77.
     \bibitem{wang09} Wang L, Bo K and Li R. Local rank inference
     for varying coefficient models. \textit{Journal of American Statistical Association} 2009; 104: 1631-1645.
     \bibitem{zhu12} Zhu H, Li R and Kong L. Multivariate varying coefficient model for functional responses. \textit{Annals of
     Statistics} 2012; 40: 2634-2666.
     \bibitem{liu13}  Liu X, Li R, Lanza ST, et al. Understanding the role of cessation fatigue in the smoking
cessation process. \textit{Drug and Alcohol Dependence} 2013; 133:
548-555.
     \bibitem{selya13} Selya AS, Dierker LC, Rose JS, et al.
Time-varying effects of smoking quantity and nicotine dependence
on adolescent smoking regularity. \textit{Drug and Alcohol
Dependence} 2013; 128: 230-237.
      \bibitem{vasilenko14} Vasilenko S, Piper M, Lanza ST, et al. Time-varying processes involved in smoking lapse in a randomized
trial of smoking cessation therapies. \textit{Nicotine & Tobacco
Research} 2014; 16S2: S135-S143.
      \bibitem{yang15} Yang H, Cranford JA, Li R, et al.
Two-stage model for time-varying effects of discrete longitudinal
covariates with applications in analysis of daily process data.
\textit{Statistics in Medicine} 2015; 34: 571-581.
      \bibitem{zucker96} Zucker RA, Ellis DA, Fitzgerald HE, et al. Other evidence for at least two alcoholisms ii: life course
variation
  in antisociality and heterogeneity of alcoholic outcome.
\textit{Development and Psychopathology} 1996; 8: 831-848.
      \bibitem{ryff10} Ryff CD and Almeida DM. National survey of
      Midlife in the United States (MIDUS II): Daily Stress
      Project, 2004-2009. ICPSR26841-v1. Ann Arbor, MI:
      Inter-university Cosortium for Political and Social Research
      [distributor], 2010-02-26.
      http://doi.org/10.3886/ICPSR26841.v1
      \bibitem{mccullagh89} McCullagh P and Nelder JA. \textit{Generalized linear models.} London: Chapman and Hall, 1989.
      \bibitem{deboor78} de Boor C. \textit{A practical guide to splines.} New York, NY: Springer-Verlag, 1978.
      \bibitem{shiyko12} Shiyko MP, Lanza ST, Tan X, et al. Using the time-varying effect model (TVEM)
        to examine dynamic associations between negative affect and self-confidence on smoking urges:
        Differences between successful quitters and relapsers. \textit{Prevention Science} 2012; 13: 288-299.
      \bibitem{simpson05} Simpson TL, Kivlahan DR, Bush KR, et al. Telephone self-monitoring among alcohol use disorder patients in
early recovery: A randomized study of feasibility and measurement
reactivity. \textit{Drug and Alcohol Dependence} 2005; 79:
241-250.
       \bibitem{tucker12} Tucker JA, Blum ER, Xie L, et al.
       Interactive voice response self-monitoring to assess risk behaviors in rural substance users living with HIV/AIDS. \textit{AIDS and Behavior} 2012; 16: 432-440.
       \bibitem{barta12} Barta WD, Tennen H and Litt MD. Measurement reactivity in diary research. In: Mehl MR and Conner TS (eds) \textit{Handbook of research methods for studying daily
       life.} New York: Guilford Press, 2012, pp.108-123.
       \bibitem{stritzke05} Stritzke WGK, Dandy J, Durkin K, et al.
       Use of interactive voice response (IVR) technology in health research with children. \textit{Behavior Research Methods} 2005; 37: 119-126.
       \bibitem{buu11} Buu A, Johnson NJ, Li R, et al. New variable selection methods for zero-inflated
       count data with applications to the substance abuse field.  \textit{Statistics in Medicine} 2011; 30: 2326-2340.
       \bibitem{buu12} Buu A, Li R, Tan X, et al. Statistical models for
       longitudinal zero-inflated count data with applications to the substance abuse field. \textit{Statistics in Medicine} 2012; 31: 4074-4086.
\end{thebibliography}


%back-end
\end{document}
